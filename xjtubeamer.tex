% !TEX program = xelatex
% !BIB program = biber

% 【注意】xcolor=table 必须在这里启用,用于支持彩色表格
\documentclass[10pt, aspectratio=169, mathserif, xcolor=table]{beamer}

% 引入我们自定义的样式文件
\usepackage{xjtubeamer}

% 参考文献设置 (使用 biblatex IEEE 风格)
\usepackage[style=ieee, backend=biber]{biblatex}
\addbibresource{reference.bib} % 导入 reference.bib

%----------------------------------------------------------------------------------------
%   用户配置区
%----------------------------------------------------------------------------------------
\title[XJTU Report]{电气工程学术报告}
\subtitle{——最终集成演示版}
\author[你的名字]{汇报人:你的名字}
\institute[XJTU]{西安交通大学 \\ 电气工程学院}
\date{\today}

% 若想关闭每章开头的目录页,请取消下面这行的注释
% \ShowSeparatorfalse 

%----------------------------------------------------------------------------------------
%   正文开始
%----------------------------------------------------------------------------------------
\begin{document}

% 1. 封面
\begin{frame}
    \titlepage
\end{frame}

% 2. 总目录
\begin{frame}{\textbf{目录}}
    \tableofcontents[hideallsubsections]
\end{frame}

% --- 第1章:表格与布局 ---
\section{表格与排版}

\begin{frame}{自适应宽表格 (Tabularx)}
    使用我们定义的 \texttt{tabularx} 环境,表格宽度设为页面 \textbf{95\%}。
    
    \begin{table}[H]
        \centering
        % 隔行变色设置
        \rowcolors{2}{RowColorOdd}{RowColorEven}
        \caption{PMSM 电机参数表 (宽度自适应)}
        
        \begin{tabularx}{0.95\textwidth}{Y Y Y Y} 
            % 表头:深蓝底白字
            \rowcolor{XJTUBlue} \color{white}\textbf{参数名称} & \color{white}\textbf{符号} & \color{white}\textbf{数值} & \color{white}\textbf{单位} \\
            额定功率 & $P_N$ & 3.0 & kW \\
            额定电压 & $U_N$ & 380 & V \\
            额定转速 & $n_N$ & 1500 & r/min \\
            定子电阻 & $R_s$ & 0.85 & $\Omega$ \\
            直轴电感 & $L_d$ & 4.5 & mH \\
        \end{tabularx}
    \end{table}
\end{frame}

% --- 第2章:代码演示 ---
\section{代码演示}

\begin{frame}[fragile]{MATLAB 专用环境}
    使用 \texttt{matlabcode} 环境,配色更接近 MATLAB 编辑器。
    
    \begin{matlabcode}
% 空间矢量脉宽调制 (SVPWM)
if (T1 + T2 > Ts)
    % 过调制处理
    ratio = Ts / (T1 + T2);
    T1 = T1 * ratio;
    T2 = T2 * ratio;
end
T0 = Ts - T1 - T2;
    \end{matlabcode}
\end{frame}

\begin{frame}[fragile]{Python/C++ 通用环境}
    使用标准的 \texttt{lstlisting} 环境,适合展示算法逻辑。
    
    \begin{block}{Python 快速排序}
    \begin{lstlisting}[language=Python, basicstyle=\ttfamily\small, keywordstyle=\color{blue!80!black}, commentstyle=\color{green!60!black}, frame=single]
def quick_sort(arr):
    if len(arr) <= 1:
        return arr
    pivot = arr[len(arr) // 2]
    # 列表推导式
    left = [x for x in arr if x < pivot]
    middle = [x for x in arr if x == pivot]
    right = [x for x in arr if x > pivot]
    return quick_sort(left) + middle + quick_sort(right)
    \end{lstlisting}
    \end{block}
\end{frame}

% --- 第3章:参考文献 ---
\section{文献引用}

\begin{frame}{学术引用演示}
    \begin{itemize}
        \item \textbf{教材引用}:
        关于 LaTeX 的经典参考书 \cite{lamport1994latex}。
        
        \item \textbf{期刊论文}:
        信息论的基础是香农提出的 \cite{shannon1948mathematical}。
        
        \item \textbf{会议论文}:
        深度学习中的 Attention 机制 \cite{vaswani2017attention}。
    \end{itemize}
    
    \vspace{1cm}
    \begin{alertblock}{说明}
        这些引用会自动生成下方的文献列表。编译顺序:XeLaTeX -> Biber -> XeLaTeX。
    \end{alertblock}
\end{frame}

% 文献列表页
\begin{frame}[allowframebreaks]{参考文献}
    \printbibliography
\end{frame}

% --- 结束页 ---
\section{总结}
\begin{frame}
    \centering
    \Huge \textcolor{XJTUBlue}{谢\quad 谢!}
    
    \vspace{1cm}
    \large 敬请各位专家批评指正
\end{frame}

\end{document}